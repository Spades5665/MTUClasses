\documentclass{article}
\usepackage{graphicx} % Required for inserting images
\usepackage{amsfonts}

\title{Newton's Method}
\author{David Cunningham, Josie Towery, Caden Wilfong}
\date{October 2023}


\begin{document}


\begin{abstract}
   Newton's Method abstract 
\end{abstract}


\maketitle


\section{Introduction}
    \begin{itemize}
        \item \cite{encyclopedianewton} The purpose of Newton's Method is to find the roots of a function when given a value of x. The method finds the closest root to the given x by subtracting the the value of the function at x divided by the value of the derivative of the function at x. This process is repeated over and over until the change in the new calculated x and the x used is minuscule.  
    \end{itemize} 

\section{Newton's Method Formula}

\begin{itemize}
    \item $f(x): \mathbb{R}^n \rightarrow \mathbb{R}^n$ A function of x
    \item $df(x): \mathbb{R}^n \rightarrow \mathbb{R}^{n \times n}$ Derivative of a function f of x
\end{itemize}

\begin{equation}
    x_{k+1} = x_k - linsol(df(x), f(x))
\end{equation}


\section{Under Determined Newton's Method}
\begin{itemize}
    \item What is this? 
    \item Why does it still work (is this unexpected)?
    \item Are there limits to where Newton's Method can work?
\end{itemize}

Newton's method \cite{atkinson1991introduction}can also be applied to an undetermined system of linear equations. An undetermined system is one with more equations than variables. 


\section{Conclusion}
\begin{itemize}
    \item Review of Newton's Method
    \item Expansion to under determined systems
    \item Final thoughts?
\end{itemize}


\bibliographystyle{alphadin}
%\bibliographystyle{IEEEtranN}
\bibliography{refs}


\end{document}
